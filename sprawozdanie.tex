\documentclass[polish,a4paper,11pt]{mwart}

\usepackage[polish]{babel}
\usepackage[utf8]{inputenc}
\usepackage{polski}
\usepackage[T1]{fontenc}
\usepackage{lmodern}  % zestaw fontów
\usepackage{indentfirst}
\frenchspacing

\usepackage{enumerate}
\usepackage{graphicx}
\usepackage{float}
\usepackage{makecell}
\usepackage{siunitx}
\sisetup{output-decimal-marker = {,}}
\usepackage{icomma}
\let\lll\undefined
\usepackage{amsmath, amssymb, amsfonts}
\usepackage{mathtools}
\usepackage{import}		% wklejanie pdf_tex
\usepackage{xcolor}		% kolory
\usepackage{multirow}
\usepackage{microtype}
\usepackage{tabularx}
\usepackage{pgfplots}

\usepackage{csquotes}
\DeclareQuoteAlias{croatian}{polish}

\usepackage{placeins}	% poprawia float

\let\Oldsection\section
\renewcommand{\section}{\FloatBarrier\Oldsection}

\let\Oldsubsection\subsection
\renewcommand{\subsection}{\FloatBarrier\Oldsubsection}

\let\Oldsubsubsection\subsubsection
\renewcommand{\subsubsection}{\FloatBarrier\Oldsubsubsection}

\AtBeginDocument{
  \renewcommand{\tablename}{Tab.}
  \renewcommand{\figurename}{Rys.}
}

\setlength{\emergencystretch}{3em}

\DeclareSIUnit\decibelV{dBV}

\date{\today}
\author{Zuzanna Kusal, Szymon Mikulicz, Marcel Piszak, Anna Warowny}
\title{Sprawozdanie z projektu redukcji drgań płyty}

\begin{document}

\maketitle

\section{Cel}

Celem projektu było zaprojektowanie, zamodelowanie i zbudowanie podzespołu
układu do redukcji drgań płyty kwadratowej, jednostronnie utwierdzonej,
pobudzonej do drgań w częstotliwościach własnych, a następnie przeprowadzenie
redukcji dla 5 częstotliwości własnych drgań wywołanych wzbudnikiem, z
wykorzystaniem 1 lub 2 aktuatorów względem 1 lub 2 czujników (9 konfiguracji
dla każdej częstotliwości).

\section{Identyfikacja częstotliwości własnych}

W programie Salome Meca zamodelowano płytę kwadratową o wymiarach \SI{250 x 250}{\milli\meter}.
Zbadano jej mody oraz wybrano pięć, względem których w dalszej części
przeprowadzono redukcję drgań. Są to częstotliwości: 278, 313, 549, 727 i
\SI{947}{\hertz}. Postacie tych modów przedstawiono na rys. \ref{fig:mod1},
\ref{fig:mod2}, \ref{fig:mod3}, \ref{fig:mod4} i \ref{fig:mod5}.

\begin{figure}[!tbh]
  \centering
  \includegraphics[width=\textwidth]{./plate_vib/278Hz.png}
  \caption{Postać modu o częstotliwości \SI{278}{\hertz}}
  \label{fig:mod1}
\end{figure}

\begin{figure}[!tbh]
  \centering
  \includegraphics[width=\textwidth]{./plate_vib/313Hz.png}
  \caption{Postać modu o częstotliwości \SI{313}{\hertz}}
  \label{fig:mod2}
\end{figure}

\begin{figure}[!tbh]
  \centering
  \includegraphics[width=\textwidth]{./plate_vib/549Hz.png}
  \caption{Postać modu o częstotliwości \SI{549}{\hertz}}
  \label{fig:mod3}
\end{figure}

\begin{figure}[!tbh]
  \centering
  \includegraphics[width=\textwidth]{./plate_vib/727Hz.png}
  \caption{Postać modu o częstotliwości \SI{727}{\hertz}}
  \label{fig:mod4}
\end{figure}

\begin{figure}[!tbh]
  \centering
  \includegraphics[width=\textwidth]{./plate_vib/947Hz.png}
  \caption{Postać modu o częstotliwości \SI{947}{\hertz}}
  \label{fig:mod5}
\end{figure}

\section{Budowa układu redukcji drgań płyty}

Na początku zaprojektowano podzespół układu do redukcji drgań płyty. Schemat
całego układu przedstawiono na rys. \ref{fig:schemat}

\begin{figure}[!tbh]
  \centering
  \import{./vecgraphics/}{PMRD_plate.pdf_tex}
  \caption{Schemat układu pomiarowego}
  \label{fig:schemat}
\end{figure}

Zbudowano część zawierającą płytę, elementy piezoelektryczne i~przewody
ostatecznie zakończone złączami BNC, które podłączono do wzmacniacza i~karty
pomiarowej.  Płyta jest aluminiowa, o wymiarach \SI{250 x 270}{\milli\meter}
(\SI{20}{\milli\meter}
przeznaczone na przymocowanie do imadła). Elementy piezoelektryczne
przymocowano do płyty wykorzystując klej dwuskładnikowy do metalu. Między
płytą, a elementami piezoelektrycznymi umieszczono miedziane płytki, na styk, o
grubości \SI{0.1}{\milli\meter} odpowiadające za przewodzenie sygnału elektrycznego między
wewnętrzną stroną piezoelektryka, a przewodem, do tej płytki przylutowanym.
Drugi przewód przylutowano do zewnętrznej strony piezoelektryka. Rozmieszczenie
elementów piezoelektrycznych (wzbudnika W, czujników $S$ i aktuatorów $A$) oraz
sposób ich przymocowania przedstawiono na rys. \ref{fig:plate1} i
\ref{fig:plate2}.

\begin{figure}[!tbh]
  \centering
  \includegraphics[width=\textwidth]{./bitgraphics/plate1.jpg}
  \caption{Widok na wzbudnik oraz czujniki zainstalowane na płycie}
  \label{fig:plate1}
\end{figure}

\begin{figure}[!tbh]
  \centering
  \includegraphics[width=\textwidth]{./bitgraphics/plate2.jpg}
  \caption{Widok na aktuatory zainstalowane na płycie}
  \label{fig:plate2}
\end{figure}

Drugie końce przewodów przymocowano do kostki elektrycznej, na której wyjściu
przyłączono przewody zakończone złączami BNC.

\section{Przeprowadzenie redukcji drgań płyty}

Podzespół połączono z pozostałymi elementami układu i w pierwszej kolejności
zbadano mody płyty (z elementami piezoelektrycznymi). Wykorzystano w tym celu
program napisany w środowisku programistycznym LabVIEW. Zaobserwowano wzrost
amplitudy między innymi dla częstotliwości 250, 300, 570, 710 oraz
\SI{920}{\hertz} odpowiadające wybranym przez nas częstotliwościom, których
mody przedstawiono na rys. \ref{fig:mod1}--\ref{fig:mod5}. Pełny rozkład
amplitud dla czujnika $S1$ oraz czujnika $S2$ w zależności od wzbudzonej
częstotliwości z zakresu od 10 do \SI{2000}{\hertz}, dla stałej amplitudy
pobudzenia przedstawiono na rys, \ref{fig:mody}.

\begin{figure}[!tbh]
  \input{plots/mody.pgf}
  \caption{Rozkład amplitud dla czujników}
  \label{fig:mody}
\end{figure}

Widać wyraźnie niższe amplitudy dla czujnika 2. Pomimo tego zdecydowano się na
kontynuację redukcji drgań płyty.

W kolejnym kroku z wykorzystaniem programu napisanego w środowisku
programistycznym LabVIEW\texttrademark dla każdej z powyżej wymienionej częstotliwości przy
stałym pobudzeniu \SI{20}{\volt} dążono do odnalezienia minimalnej amplitudy dla
czujnika $S1$, czujnika $S2$ lub sumy z tych czujników zmieniając amplitudę oraz fazę
aktuatora $A1$, aktuatora $A2$ lub obydwu aktuatorów (każdy względem każdego, 9
konfiguracji dla każdej częstotliwości).

\section{Wyniki}

Wyniki redukcji drgań dla kolejnych częstotliwości przedstawiono w tab.
\ref{tab:red1}--\ref{tab:red5}. Wartości amplitud zadanych na aktuatory są to
wartości wprowadzane do programu, przed wzmocnieniem, rzeczywiste wartości
napięcia podane na aktuatory były dziesięciokrotnie większe. Na czerwono w
tablicach zaznaczono sytuacje w których nie udało się uzyskań napięcia
potrzebnego do redukcji z~powodu niewystarczającej mocy wzmacniacza.

\begin{table}[!tbh]
  \centering
  \caption{Wyniki redukcji drgań dla modu \SI{250}{\hertz}}
  \label{tab:red1}
  \begin{tabular}{|c|c|c|c|c|c|c|}
    \cline{3-7}
    \multicolumn{2}{c|}{}&\multicolumn{2}{c|}{Aktuator}&\multicolumn{3}{c|}{Czujnik [\si{\decibelV}]}\\\cline{3-7}
    \multicolumn{2}{c|}{}&$A1$&$A2$&$S1$&$S2$&$S1+S2$\\\hline
    \multirow{2}{*}{tło}               &   $A [\si{\V}]$ & - & - & -84,0 & -90,2 & -83,3 \\\cline{2-7}
				       &$\Phi [\si{\degree}]$ & - & - & \multicolumn{3}{c}{}\\\hline
    \multirow{2}{*}{bez redukcji}      &   $A [\si{\V}]$ & - & - & -19,0 & -34,5 & -18,8 \\\cline{2-7}
				       &$\Phi [\si{\degree}]$ & - & - & \multicolumn{3}{c}{}\\\hline
    \multirow{6}{*}{$\min\{S1\}$}      &   $A [\si{\V}]$ & 29,2 & - & \textbf{-63,5} & -52,1 & -51,8 \\\cline{2-7}
				       &$\Phi [\si{\degree}]$ & 15 & - & \multicolumn{3}{c}{}\\\cline{2-7}
				       &   $A [\si{\V}]$ & - & 34,6 & \textbf{-62,8} & -42,0 & -42,0 \\\cline{2-7}
				       &$\Phi [\si{\degree}]$ & - & 23 & \multicolumn{3}{c}{}\\\cline{2-7}
				       &   $A [\si{\V}]$ & 29,2 & 0 & \textbf{-63,5} & -52,1 & -51,8 \\\cline{2-7}
				       &$\Phi [\si{\degree}]$ & 15 & 0 & \multicolumn{3}{c}{}\\\hline
    \multirow{6}{*}{$\min\{S2\}$}      &   $A [\si{\V}]$ & 27,8 & - & -36,0 & \textbf{-73,3} & -36,0\\\cline{2-7}
				       &$\Phi [\si{\degree}]$ & 10 & - & \multicolumn{3}{c}{}\\\cline{2-7}
				       &   $A [\si{\V}]$ & - & 31 & -26,8 & \textbf{-67,3} & -26,8 \\\cline{2-7}
				       &$\Phi [\si{\degree}]$ & - & 0 & \multicolumn{3}{c}{}\\\cline{2-7}
				       &   $A [\si{\V}]$ & 25,5 & 2,9 & -36,4 & \textbf{-80,1} & -36,4 \\\cline{2-7}
				       &$\Phi [\si{\degree}]$ & 6 & 69 & \multicolumn{3}{c}{}\\\hline
    \multirow{6}{*}{$\min\{S1+S2\}$}   &   $A [\si{\V}]$ & 29,2 & - & -63,5  & -52,1 & \textbf{-51,8}\\\cline{2-7}
				       &$\Phi [\si{\degree}]$ & 15 & - & \multicolumn{3}{c}{}\\\cline{2-7}
				       &   $A [\si{\V}]$ & - & 34,2 & -59,2 & -42,9 & \textbf{-42,8} \\\cline{2-7}
				       &$\Phi [\si{\degree}]$ & - & 24 & \multicolumn{3}{c}{}\\\cline{2-7}
				       &   $A [\si{\V}]$ & 29,2 & 0 & -63,5 & -52,1 & \textbf{-51,8} \\\cline{2-7}
				       &$\Phi [\si{\degree}]$ & 15 & 0 & \multicolumn{3}{c}{}\\\cline{1-4}
  \end{tabular}
\end{table}

\begin{table}[!tbh]
  \centering
  \caption{Wyniki redukcji drgań dla modu \SI{300}{\hertz}}
  \label{tab:red2}
  \begin{tabular}{|c|c|c|c|c|c|c|}
    \cline{3-7}
    \multicolumn{2}{c|}{}&\multicolumn{2}{c|}{Aktuator}&\multicolumn{3}{c|}{Czujnik [\si{\decibelV}]}\\\cline{3-7}
    \multicolumn{2}{c|}{}&$A1$&$A2$&$S1$&$S2$&$S1+S2$\\\hline
    \multirow{2}{*}{tło}               &   $A [\si{\V}]$ & - & - & -100,0 & -99,0 & -99,0 \\\cline{2-7}
				       &$\Phi [\si{\degree}]$ & - & - & \multicolumn{3}{c}{}\\\hline
    \multirow{2}{*}{bez redukcji}      &   $A [\si{\V}]$ & - & - & -19,9 & -40,0 & -19,9 \\\cline{2-7}
				       &$\Phi [\si{\degree}]$ & - & - & \multicolumn{3}{c}{}\\\hline
    \multirow{6}{*}{$\min\{S1\}$}      &   $A [\si{\V}]$ & \textcolor{red}{51} & - & \textbf{-29,7} & -45,0 & -21,7\\\cline{2-7}
				       &$\Phi [\si{\degree}]$ & 292 & - & \multicolumn{3}{c}{}\\\cline{2-7}
				       &   $A [\si{\V}]$ & - & \textcolor{red}{40} & \textbf{-22,0} & -43,6 & -22,0 \\\cline{2-7}
				       &$\Phi [\si{\degree}]$ & - & 180 & \multicolumn{3}{c}{}\\\cline{2-7}
				       &   $A [\si{\V}]$ & \textcolor{red}{51} & \textcolor{red}{40} & \textbf{-24,5} & -51,3 & -24,5 \\\cline{2-7}
				       &$\Phi [\si{\degree}]$ & 294 & 180 & \multicolumn{3}{c}{}\\\hline
    \multirow{6}{*}{$\min\{S2\}$}      &   $A [\si{\V}]$ & \textcolor{red}{51} & - & -29,7 & \textbf{-45,0} & -21,7 \\\cline{2-7}
				       &$\Phi [\si{\degree}]$ & 292 & - & \multicolumn{3}{c}{}\\\cline{2-7}
				       &   $A [\si{\V}]$ & - & \textcolor{red}{40} & -21,5 & \textbf{-44,6} & -21,4 \\\cline{2-7}
				       &$\Phi [\si{\degree}]$ & - & 210 & \multicolumn{3}{c}{}\\\cline{2-7}
				       &   $A [\si{\V}]$ & \textcolor{red}{51} & \textcolor{red}{40} & -23,7 & \textbf{-59,5} & -13,7\\\cline{2-7}
				       &$\Phi [\si{\degree}]$ & 291 & 211 & \multicolumn{3}{c}{}\\\hline
    \multirow{6}{*}{$\min\{S1+S2\}$}   &   $A [\si{\V}]$ & \textcolor{red}{51} & - & -29,7 & -45,0 & \textbf{-21,7}\\\cline{2-7}
				       &$\Phi [\si{\degree}]$ & 292 & - & \multicolumn{3}{c}{}\\\cline{2-7}
				       &   $A [\si{\V}]$ & - & \textcolor{red}{40} & -22,0 & -43,6 & \textbf{-22,0} \\\cline{2-7}
				       &$\Phi [\si{\degree}]$ & - & 180 & \multicolumn{3}{c}{}\\\cline{2-7}
				       &   $A [\si{\V}]$ & \textcolor{red}{51} & \textcolor{red}{40} & -24,5 & -51,3 & \textbf{-24,5} \\\cline{2-7}
				       &$\Phi [\si{\degree}]$ & 294 & 180 & \multicolumn{3}{c}{}\\\cline{1-4}
  \end{tabular}
\end{table}

\begin{table}[!tbh]
  \centering
  \caption{Wyniki redukcji drgań dla modu \SI{570}{\hertz}}
  \label{tab:red3}
  \begin{tabular}{|c|c|c|c|c|c|c|}
    \cline{3-7}
    \multicolumn{2}{c|}{}&\multicolumn{2}{c|}{Aktuator}&\multicolumn{3}{c|}{Czujnik [\si{\decibelV}]}\\\cline{3-7}
    \multicolumn{2}{c|}{}&$A1$&$A2$&$S1$&$S2$&$S1+S2$\\\hline
    \multirow{2}{*}{tło}               &   $A [\si{\V}]$ & - & - & -82,7 & -97,5 & -82,8 \\\cline{2-7}
				       &$\Phi [\si{\degree}]$ & - & - & \multicolumn{3}{c}{}\\\hline
    \multirow{2}{*}{bez redukcji}      &   $A [\si{\V}]$ & - & - & -10,2 & -31,5 & -10,2 \\\cline{2-7}
				       &$\Phi [\si{\degree}]$ & - & - & \multicolumn{3}{c}{}\\\hline
    \multirow{6}{*}{$\min\{S1\}$}      &   $A [\si{\V}]$ & 44 & - & \textbf{-54,5} & -38,0 & -37,9 \\\cline{2-7}
				       &$\Phi [\si{\degree}]$ & 26 & - & \multicolumn{3}{c}{}\\\cline{2-7}
				       &   $A [\si{\V}]$ & - & 35,7 & \textbf{-40,5} & -33,9 & -33 \\\cline{2-7}
				       &$\Phi [\si{\degree}]$ & - & 355 & \multicolumn{3}{c}{}\\\cline{2-7}
				       &   $A [\si{\V}]$ & 1,9 & 35,7 & \textbf{-65,2} & -34,0 & -34,0\\\cline{2-7}
				       &$\Phi [\si{\degree}]$ & 108 & 355 & \multicolumn{3}{c}{}\\\hline
    \multirow{6}{*}{$\min\{S2\}$}      &   $A [\si{\V}]$ & 44 & - & -17,3 & \textbf{-55,6} & -17,3 \\\cline{2-7}
				       &$\Phi [\si{\degree}]$ & 0 & - & \multicolumn{3}{c}{}\\\cline{2-7}
				       &   $A [\si{\V}]$ & - & \textcolor{red}{40} & -11,7 & \textbf{-44,1} & -11,7 \\\cline{2-7}
				       &$\Phi [\si{\degree}]$ & - & 42 & \multicolumn{3}{c}{}\\\cline{2-7}
				       &   $A [\si{\V}]$ & 44 & 3 & -15,8 & \textbf{-81,3} & -15,8 \\\cline{2-7}
				       &$\Phi [\si{\degree}]$ & 0 & 238 & \multicolumn{3}{c}{}\\\hline
    \multirow{6}{*}{$\min\{S1+S2\}$}   &   $A [\si{\V}]$ & 44 & - & -54,5 & -38,0 & \textbf{-37,9}\\\cline{2-7}
				       &$\Phi [\si{\degree}]$ & 26 & - & \multicolumn{3}{c}{}\\\cline{2-7}
				       &   $A [\si{\V}]$ & - & 35,7 & -40,5 & -33,9 & \textbf{-33} \\\cline{2-7}
				       &$\Phi [\si{\degree}]$ & - & 355 & \multicolumn{3}{c}{}\\\cline{2-7}
				       &   $A [\si{\V}]$ & 7 & 36 & -63,0 & -34,0& \textbf{-34,0} \\\cline{2-7}
				       &$\Phi [\si{\degree}]$ & 142 & 356 & \multicolumn{3}{c}{}\\\cline{1-4}
  \end{tabular}
\end{table}

\begin{table}[!tbh]
  \centering
  \caption{Wyniki redukcji drgań dla modu \SI{710}{\hertz}}
  \label{tab:red4}
  \begin{tabular}{|c|c|c|c|c|c|c|}
    \cline{3-7}
    \multicolumn{2}{c|}{}&\multicolumn{2}{c|}{Aktuator}&\multicolumn{3}{c|}{Czujnik [\si{\decibelV}]}\\\cline{3-7}
    \multicolumn{2}{c|}{}&$A1$&$A2$&$S1$&$S2$&$S1+S2$\\\hline
    \multirow{2}{*}{tło}               &   $A [\si{\V}]$ & - & - & -80,0 & -94,1 & -80,0 \\\cline{2-7}
				       &$\Phi [\si{\degree}]$ & - & - & \multicolumn{3}{c}{}\\\hline
    \multirow{2}{*}{bez redukcji}      &   $A [\si{\V}]$ & - & - & -9,5 & -29,6 & -9,4 \\\cline{2-7}
				       &$\Phi [\si{\degree}]$ & - & - & \multicolumn{3}{c}{}\\\hline
    \multirow{6}{*}{$\min\{S1\}$}      &   $A [\si{\V}]$ & 37,9 & - & \textbf{-56,3} & -52,5 & -51,0 \\\cline{2-7}
				       &$\Phi [\si{\degree}]$ & 238 & - & \multicolumn{3}{c}{}\\\cline{2-7}
				       &   $A [\si{\V}]$ & - & \textcolor{red}{40} & \textbf{-17,3} & -32,0 & -17,1 \\\cline{2-7}
				       &$\Phi [\si{\degree}]$ & - & 248 & \multicolumn{3}{c}{}\\\cline{2-7}
				       &   $A [\si{\V}]$ & 37,5 & 0,4 & \textbf{-65,4} & -48,8 & -48,8 \\\cline{2-7}
				       &$\Phi [\si{\degree}]$ & 238 & 326 & \multicolumn{3}{c}{}\\\hline
    \multirow{6}{*}{$\min\{S2\}$}      &   $A [\si{\V}]$ & 35,5 & - & -32,1 & \textbf{-82,5} & -32,1 \\\cline{2-7}
				       &$\Phi [\si{\degree}]$ & 236 & - & \multicolumn{3}{c}{}\\\cline{2-7}
				       &   $A [\si{\V}]$ & - & \textcolor{red}{39} & -16,4 & \textbf{-32,6} & -16,3 \\\cline{2-7}
				       &$\Phi [\si{\degree}]$ & - & 257 & \multicolumn{3}{c}{}\\\cline{2-7}
				       &   $A [\si{\V}]$ & 35,5 & 0 & -32,1 & \textbf{-82,5} & -32,1\\\cline{2-7}
				       &$\Phi [\si{\degree}]$ & 236 & 0 & \multicolumn{3}{c}{}\\\hline
    \multirow{6}{*}{$\min\{S1+S2\}$}   &   $A [\si{\V}]$ & 37,9 & - & -56,3 & -52,5 & \textbf{-51,0}\\\cline{2-7}
				       &$\Phi [\si{\degree}]$ & 238 & - & \multicolumn{3}{c}{}\\\cline{2-7}
				       &   $A [\si{\V}]$ & - & \textcolor{red}{40} & -17,3 & -32,0 & \textbf{-17,1} \\\cline{2-7}
				       &$\Phi [\si{\degree}]$ & - & 248 & \multicolumn{3}{c}{}\\\cline{2-7}
				       &   $A [\si{\V}]$ & 37,9 & 0,4 & -61,7 & -52,9 & \textbf{-52,3} \\\cline{2-7}
				       &$\Phi [\si{\degree}]$ & 238 & 0 & \multicolumn{3}{c}{}\\\cline{1-4}
  \end{tabular}
\end{table}

\begin{table}[!tbh]
  \centering
  \caption{Wyniki redukcji drgań dla modu \SI{920}{\hertz}}
  \label{tab:red5}
  \begin{tabular}{|c|c|c|c|c|c|c|}
    \cline{3-7}
    \multicolumn{2}{c|}{}&\multicolumn{2}{c|}{Aktuator}&\multicolumn{3}{c|}{Czujnik [\si{\decibelV}]}\\\cline{3-7}
    \multicolumn{2}{c|}{}&$A1$&$A2$&$S1$&$S2$&$S1+S2$\\\hline
    \multirow{2}{*}{tło}               &   $A [\si{\V}]$ & - & - & -89,4 & -100,0 & -89,5 \\\cline{2-7}
				       &$\Phi [\si{\degree}]$ & - & - & \multicolumn{3}{c}{}\\\hline
    \multirow{2}{*}{bez redukcji}      &   $A [\si{\V}]$ & - & - & -7,5 & -24,3 & -7,4 \\\cline{2-7}
				       &$\Phi [\si{\degree}]$ & - & - & \multicolumn{3}{c}{}\\\hline
    \multirow{6}{*}{$\min\{S1\}$}      &   $A [\si{\V}]$ & 21,4 & - & \textbf{-58,3} & -31,2 & -31,2 \\\cline{2-7}
				       &$\Phi [\si{\degree}]$ & 172 & - & \multicolumn{3}{c}{}\\\cline{2-7}
				       &   $A [\si{\V}]$ & - & 36,6 & \textbf{-62,0} & -29,3 & -29,3 \\\cline{2-7}
				       &$\Phi [\si{\degree}]$ & - & 356 & \multicolumn{3}{c}{}\\\cline{2-7}
				       &   $A [\si{\V}]$ & 1,7 & 36,6 & \textbf{-62,1} & -29,4 & -29,4 \\\cline{2-7}
				       &$\Phi [\si{\degree}]$ & 85 & 0 & \multicolumn{3}{c}{}\\\hline
    \multirow{6}{*}{$\min\{S2\}$}      &   $A [\si{\V}]$ & 35,8 & - & -9,8 & \textbf{-74,9} & -9,8\\\cline{2-7}
				       &$\Phi [\si{\degree}]$ & 184 & - & \multicolumn{3}{c}{}\\\cline{2-7}
				       &   $A [\si{\V}]$ & - & \textcolor{red}{40} & -12,4 & \textbf{-35,6} & -12,4 \\\cline{2-7}
				       &$\Phi [\si{\degree}]$ & - & 26 & \multicolumn{3}{c}{}\\\cline{2-7}
				       &   $A [\si{\V}]$ & 35,7 & 0,4 & -9,7 & \textbf{-77,4} & -9,7\\\cline{2-7}
				       &$\Phi [\si{\degree}]$ & 184 & 332 & \multicolumn{3}{c}{}\\\hline
    \multirow{6}{*}{$\min\{S1+S2\}$}   &   $A [\si{\V}]$ & 21,4 & - & -58,3 & -31,2 & \textbf{-31,2} \\\cline{2-7}
				       &$\Phi [\si{\degree}]$ & 172 & - & \multicolumn{3}{c}{}\\\cline{2-7}
				       &   $A [\si{\V}]$ & - & 36,6 & -62,0 & -29,3 & \textbf{-29,3} \\\cline{2-7}
				       &$\Phi [\si{\degree}]$ & - & 356 & \multicolumn{3}{c}{}\\\cline{2-7}
				       &   $A [\si{\V}]$ & 21,3 & 0,2 & -48,9 & -31,2 & \textbf{-31,1} \\\cline{2-7}
				       &$\Phi [\si{\degree}]$ & 173 & 320 & \multicolumn{3}{c}{}\\\cline{1-4}
  \end{tabular}
\end{table}

\section{Wnioski}

Przeprowadzono redukcję drgań płyty dla pięciu częstotliwości. Nie dla każdej
częstotliwości możliwe było całkowite wytłumienie drgań względem wybranych
czujników. 

Dla częstotliwości \SI{250}{\hertz} skutecznie całkowicie wytłumiono drgania
względem czujnika $S2$ z wykorzystaniem dwóch aktuatorów. Dobry efekt też dała
modulacja jedynie aktuatorem $A1$. Względem pojedynczego czujnika
maksymalnie udało się stłumić amplitudę drgań o ok \SI{45}{\decibelV}, względem
sumy czujników o ok \SI{33}{\decibelV}.

Dla częstotliwości \SI{300}{\hertz} nie udało się uzyskać całkowitego
wytłumienia. Za niska była maksymalna wartość amplitudy, jaką można było zadać
z~aktuatorów.  Dla czujnika $S1$ maksymalnie udało się stłumić o ok.
\SI{10}{\decibelV}, dla $S2$ o ok. \SI{20}{\decibelV}, a dla sumy
\SI{5}{\decibelV}.

Dla częstotliwości \SI{570}{\hertz} podobnie jak dla \SI{250}{\hertz}
skutecznie całkowicie wytłumiono drgania względem czujnika $S2$ z wykorzystaniem
dwóch aktuatorów. Dla czujnika $S1$ maksymalnie udało się stłumić o ok.
\SI{55}{\decibelV}, dla $S2$ o ok. \SI{50}{\decibelV}, a dla sumy
\SI{28}{\decibelV}.

Dla częstotliwości \SI{710}{\hertz} w większości przypadków za niska była
maksymalna wartość amplitudy, jaką można było zadać z aktuatora $A2$.
Podobnie jak dla 250 oraz \SI{570}{\hertz} skutecznie całkowicie wytłumiono
drgania względem czujnika $S2$ z wykorzystaniem dwóch aktuatorów, a nawet samego
aktuatora $A1$. Dla czujnika $S1$ maksymalnie udało się stłumić
amplitudę drgań o ok. \SI{55}{\decibelV}, dla $S2$ o ok.
\SI{50}{\decibelV}, a dla sumy \SI{43}{\decibelV}.

Dla częstotliwości \SI{920}{\hertz} w jednym przypadku  za niska była
maksymalna wartość amplitudy, jaką można było zadać z aktuatora $A2$. Dla
czujnika $S1$ maksymalnie udało się stłumić amplitudę drgań o ok.
\SI{55}{\decibelV}, dla $S2$ o ok.  \SI{53}{\decibelV}, a dla sumy
\SI{25}{\decibelV}.

Niższa amplituda drgań dla czujnika $S2$ mogła wynikać z nieodpowiedniego
przyklejenia czujnika do płyty (za mało kleju, nieodpowiednio rozprowadzony, za
mała powierzchnia styku z płytą) lub z uszkodzenia elementu piezoelektrycznego
(nie został on sprawdzony przed zamontowaniem). Nieznane są inne możliwe
przyczyny, jednak nie wykluczone jest ich występowanie.

\end{document}
